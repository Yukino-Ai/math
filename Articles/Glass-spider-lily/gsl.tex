% ❄ Preamble ❄
\documentclass{article}
\usepackage{black-snow}

% ❄ Document ❄
\begin{document}
\title{Glass spider lily\\
	\large Backround reading on Ramsey theory}
\author{Ai Yukino}
\date{Created: December 25, 2021\\
	\large Updated: December 28, 2021}
\maketitle
\hypersetup{linkcolor = internallinkcolor}
\tableofcontents
\hypersetup{linkcolor= .}

\section{Fundamental theorem of arithemtic}

A simple way to state the fundamental theorem of arithmetic is

\begin{theorem}[Fundamental theorem of aritmetic]\label{thm:arithmetic}
	Every natural number has a unique prime decomposition.
\end{theorem}

\noindent Then we can add some definitions to clarify how we might prove this result.

\begin{definition}
	The \wordColor{natural numbers} is the set
	\begin{equation*}
		\color{ibis}{\bmColor{N} \defined \{1, 2, \ldots\}}\color{sakura}{.}
	\end{equation*}
	Let $n\in\bmColor{N}$.
	We say a finite set of natural numbers
	\begin{equation*}
		\color{ibis}{\{n_1, n_2, \ldots, n_k\}} \defined \color{green}{S}
	\end{equation*}
	is a \wordColor{decomposition} of $n$ if their product forms $n$, i.e.
	\begin{equation*}
		\color{ibis}{n = n_1 n_2 \cdots n_k}\color{sakura}{.}
	\end{equation*}
	In such a case, each $\color{ibis}{n_i}\in \color{green}{S}$ is called a \wordColor{component} of the decomposition $\color{green}{S}$ of $\color{ibis}{n}$.
	If $\color{ibis}{p\neq 1}$ is a natural number such that
	\begin{equation*}
		\color{ibis}{\{1, p\}}
	\end{equation*}
	is its only decompsition, then $\color{ibis}{p}$ is called a \wordColor{prime number} or just a \wordColor{prime}.
	So finally, $\color{ibis}{n}$ has a \wordColor{prime decomposition} if it has a decompsition
	\begin{equation*}
		\color{ibis}{\{p_1, p_2, \ldots, p_j\}} \defined \color{green}{S'}
	\end{equation*}
	consisting solely of prime componernts.
\end{definition}
\noindent Normally people refer to the ``components'' in the above definition as \wordColor{factors} of $\color{ibis}{n}$ or more carefully define \wordColor{divisors} $\color{ibis}{m}$ of $\color{ibis}{n}$, i.e.
\begin{equation*}
	\color{ibis}{n = m l}
\end{equation*}
for some $\color{ibis}{l}\in\bmColor{N}$.
Regardless, Theorem \refColor{\ref{thm:arithmetic}} suggests (maybe misleadingly) that you can learn a lot about natural numbers by just studying natural numbers themselves.
All of this said, it's not obvious to me how to prove this result.
For now, it may be fruitful to continue on the main focus of this article.

\subsection{What about Ramsey theory?}

So does this result have anything to do with Ramsey theory?
The Wikipedia article on Ramsey theory \cite{wiki:ramsey2021} (at least when accesed on 12-28-21), says

\begin{displayquote}
	A typical result in Ramsey theory starts with some mathematical structure that is then cut into pieces. How big must the original structure be in order to ensure that at least one of the pieces has a given interesting property? This idea can be defined as \href{https://en.wikipedia.org/wiki/Partition_regularity}{partition regularity}.
\end{displayquote}

\noindent Perhaps then it would be instructive to restate Theorem \refColor{\ref{thm:arithmetic}} and/or its related definitions in terms of partition regularity.

\subsection{Euclid's theorem}

\section{Ramsey's theorem}

\section{Van der Waerden's theorem}

\bibliographystyle{plain}
\bibliography{gsl}

\end{document}