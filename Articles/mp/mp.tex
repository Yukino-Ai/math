\documentclass{article}

\usepackage{mp}

%%%%%%%%%%%%%%%%%%%%%%%%%%%%%%%%%%%%%%%%%%%%%%%%%%%%%%%%%%%%%%%%%%%%%%%%%%

\begin{document}
\title{Midnight magpie \\
	\large Mathematically rigorous notes on basic statistics}
\author{Ai Yukino}
\date{January 23, 2022}
\maketitle
\hypersetup{linkcolor = internallinkcolor}
\tableofcontents
\hypersetup{linkcolor= .}

\section{Organization of statistics}

\href{https://en.wikipedia.org/wiki/Outline_of_statistics}{Outline of statistics | Wikipedia}

\subsection{Statistical inference}

Given random variables sampled from an underlying population with an unkown probability distribution (could be fully parametric, non-parametric, and semi-parametric), what can we say about this distribution?

\subsection{Descriptive statistics}

Given random variables which are not assumed to come from an underlying population, what can we say about these random variables themselves?
(For example, the standard deviation $\sigma$ is a descriptives statistic.)
\section{Mathematical foundations for hypothesis testing}

Hypothesis testing (p-values and confidence intervals) $\dots \leftarrow$ central limit theorems $\dots \leftarrow$ statistical inference

% \noindent\faGlobe\space\href{https://vitejs.dev/}{General link}\\
% \noindent\faGithub\space\href{https://github.com/}{GitHub link}

\end{document}